\documentclass[a4paper, 12pt]{article}
\usepackage{graphicx} % Required for inserting images
\usepackage{fullpage}
\usepackage{amsmath}
\usepackage{xcolor}
\usepackage{float}
\usepackage{geometry}
\usepackage{biblatex}
\geometry{margin=1in}
\usepackage{enumitem}
\usepackage{hyperref}
\usepackage{microtype}
\usepackage{parskip}

\hypersetup{
    colorlinks=true,        % Enable colored links
    linkcolor=teal,         % Set color for internal links
    citecolor=teal,         % Set color for citations
    filecolor=teal,         % Set color for file links
    urlcolor=teal           % Set color for URLs
}

\title{04-13-25 Area and Perimeter of Squares, Rectangles, and Triangles}
\author{C\&L Math Tutoring}
\date{April 2025}

\begin{document}

\maketitle

\subsection*{Rectangles}
The formula for the area of a rectangle is:

$$A = l \times w $$

and the formula for the perimeter of a rectangle is:

$$P = (l \times 2) + (w \times 2)$$

where $l$ is the length and $w$ is the width.

Example: Francis is planting a big field of roses for Ellen in a rectangular plot. One side of the field measures 5 meters and the other measures 9 meters. Find the area and perimeter of the plot.

Solution: To find the area, use the area formula.

\begin{gather*}
A = l \times w \\
A = 9 \: m \times 5 \:  m \\
A = \boxed{45 \: m^2}
\end{gather*}

To find the perimeter:

\begin{gather*}
P = (l \times 2) + (w \times 2) \\
P = (9 \: m \times 2) + (5 \: m \times 2) \\
P = \boxed{28 \: m}
\end{gather*}

\subsection*{Squares}
Squares are rectangles, so you can use the rectangle formulas as well. However, these two formulas are unique to squares.

Area formula of a square:

$$A = s^2$$

Perimeter formula of a square:

$$P = 4 \times s$$

where $s$ is one side.

Example: Nate, while being chased by a cat, is running in a perfect square. The area of the square is 49 $m^2$. What is the side length of the square, in centimeters?

Solution: Use the area formula.

\begin{gather*}
A = s^2 \\
49 = s^2 \\
s = \sqrt{49 \: m^2} \\
s = 7 \: m \\
(7 \: m)\left(\frac{100 \: cm}{1 \: m}\right) = \boxed{700 \: cm}
\end{gather*}

\subsection*{Triangles}
The area formula for a triangle is as follows:

$$A = \frac{1}{2} \times b \times h$$

where $b$ is the base and $h$ is the height.

Example: Nate has arranged flowers for Jenny in a triangle with a base measuring 50 $cm$ and a height measuring 20 $cm$. What is the area of the triangle filled by the flowers?

\begin{gather*}
A = \frac{1}{2} \times b \times h \\
A = \frac{1}{2} \times 50 \: cm \times 20 \: cm \\
A = \frac{1}{2} \times 1000 \: cm^2 \\ 
A = \boxed{500 \: cm^2}
\end{gather*}

\huge{Make sure to be careful of the units!}

\end{document}
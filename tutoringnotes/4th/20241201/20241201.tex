\documentclass[a4paper, 12pt]{article}
\usepackage{fullpage}
\usepackage{amsmath}
\usepackage{xcolor}
\usepackage{graphicx} % Required for inserting images
\usepackage{parskip}

\title{12-1-24 Multiplying and Dividing Fractions}
\author{C\&L Math Tutoring}
\date{December 1, 2024}

\begin{document}

\maketitle

\subsection*{LCM and GCF}
\subsubsection*{Least Common Multiple}
The \textbf{least common multiple} (LCM) of a group of numbers is the smallest number that is divisible by each number in the group. For example, the LCM of 3, 5, and 10 is 30, because 30 is the smallest number that is divisible by 3, 5, and 10.

\subsubsection*{Greatest Common Factor}
The \textbf{greatest common factor} (GCF) of a group of numbers is the largest number that each number in the group is divisible by. For example, the GCF of 12, 24, and 6 is 6, because 6 is the largest number that 12, 24, and 6 are all divisible by.

\subsection*{Multiplying Fractions}
To multiply fractions, multiply the numerators and denominators together.

Example:
$$\frac{3}{5} \times \frac{5}{9} = \frac{3 \times5}{5 \times 9}$$
\\
$$=\frac{15}{45}$$
\\
$$=\boxed{{\frac{1}{3}}}\footnote{Don't forget to simplify using the GCF!}$$

\subsection*{Dividing Fractions}
To divide fractions, multiply the first fraction by the \textbf{reciprocal} of the second. (In the words of my algebra teacher, keep it, switch it, flip it!)

Example:
$$\frac{2}{7} \div \frac{5}{14}$$

\textcolor{blue}{\textbf{Step 1:} Keep it:} Keep the first fraction.
$$\frac{2}{7} \div \frac{5}{14}$$

\textcolor{blue}{\textbf{Step 2:} Switch it:} Switch the sign from division to multiplication.
$$\frac{2}{7} \times \frac{5}{14}$$

\textcolor{blue}{\textbf{Step 3:} Flip it:} Flip the second fraction.
$$\frac{2}{7} \times \frac{14}{5}$$

\textcolor{blue}{\textbf{Step 4:} Solve:} Now, apply the rules of multiplication.
$$\frac{2}{7} \times \frac{14}{5} = \frac{28}{35}$$
\\
$$=\boxed{\frac{4}{5}}$$



\end{document}

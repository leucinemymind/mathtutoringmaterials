\documentclass[a4paper, 12pt]{article}
\usepackage{graphicx} % Required for inserting images
\usepackage{textcomp} % for.... something? it got rid of an error
\usepackage{fullpage} % fill the page
\usepackage{amsmath, amsthm, amssymb} % math(s)
\usepackage{xcolor} %colored text
\usepackage{float} %float placement
\usepackage{geometry} % margin adjustment
\geometry{margin=1in}
\usepackage{biblatex} % for bibliographies
\usepackage{enumitem} % changing list appearance
\usepackage{hyperref} % outside references
\usepackage{microtype} % also got rid of an error
\usepackage{gensymb} % for degree symb?
\usepackage{parskip} % automatic line skip
\usepackage{tikz} % diagrams!
\usepackage{caption}
\hypersetup{
    colorlinks=true,        % Enable colored links
    linkcolor=teal,         % Set color for internal links
    citecolor=teal,         % Set color for citations
    filecolor=teal,         % Set color for file links
    urlcolor=teal           % Set color for URLs
}

\title{03-30-2025 Polygons}
\author{C\&L Math Tutoring}
\date{March 30, 2025}

\begin{document}

\maketitle

\subsection*{Types of polygons}

A \textbf{polygon} is a closed figure made up of straight lines in a two-dimensional plane. Polygons are named according to the number of sides they have.

They are named as follows:

\begin{enumerate}
	\item n/a
	\item n/a
	\item trigon (triangle), 180 degrees
	\begin{itemize}
	\item equiangular equilateral (60 degree angles)
	\item right isosceles (45-45-90)
	\item acute isosceles
	\item obtuse isosceles
	\item right scalene
	\item acute scalene
	\item obtuse scalene
	\end{itemize}
	\item quadrilateral, 360 degrees
	\begin{itemize}
	\item kites
	\item parallelogram
	\begin{itemize}
	\item square
	\item rectangle
	\item rhombus
	\end{itemize}
	\item trapezoid
	\end{itemize}
	\item pentagon, 540 degrees
	\item hexagon, 720 degrees
	\item heptagon, 900 degrees
	\item octagon, 1080 degrees
	\item nonagon, 1260 degrees
	\item decagon, 1440 degrees
	\item hendecagon, 1620 degrees
	\item dodecagon, 1800 degrees
\end{enumerate}

\subsection*{Sum of angle measures}
The sum of angle measures is determined by this formula:

$$S = 180 \times (n-2)$$

where:
\begin{itemize}
\item $n$ is the number of sides
\item $S$ is the sum of angle measures
\end{itemize}

\end{document}
\documentclass[a4paper, 12pt]{article}
\usepackage{graphicx} % Required for inserting images
\usepackage{fullpage}
\usepackage{amsmath}
\usepackage{xcolor}
\usepackage{float}
\usepackage{geometry}
\usepackage{biblatex}
\geometry{margin=1in}
\usepackage{enumitem}
\usepackage{hyperref}
\usepackage{microtype}
\usepackage{parskip}

\hypersetup{
    colorlinks=true,        % Enable colored links
    linkcolor=teal,         % Set color for internal links
    citecolor=teal,         % Set color for citations
    filecolor=teal,         % Set color for file links
    urlcolor=teal           % Set color for URLs
}

\title{05-11-25 Fraction-Decimal Conversions}
\author{C\&L Math Tutoring}
\date{May 2025}

\begin{document}

\maketitle

\subsection*{Review of decimal place values}

$$3.141592653589$$

3: ones

(all place values after the decimal point ``.'' represent a number $<1$)

1: tenths

4: hundredths

1: thousandths

5: ten thousandths

9: hundred thousandths

2: millionths

6: ten millionths

5: hundred millionths

3: billionths

5: ten billionths

8: hundred billionths

9: trillionths

Example: How would you write this decimal using place value names? 0.578

Solution: Five hundred seventy-eight thousandths

\subsection*{Decimals to fractions}

Example: Convert 0.578 to word form, then into a fraction.

$$0.578 = \text{five hundred seventy-eight thousandths} = \frac{578}{1000}$$

Simplify:

$$\frac{578 \div 2}{1000 \div 2} = \frac{289}{500}$$

Solution:

$$0.578 = \text{five hundred seventy-eight thousandths} = \frac{289}{500}$$

\subsection*{Fractions to decimals}

Example: Convert $\frac{3}{25}$ to a decimal.

You could just divide 3 by 25, but there is a much easier way without having to do long division, which is to put the numerator over 100.

$$\frac{3 \times 4}{25 \times 4} = \frac{12}{100}$$

This number is pronounced as ``twelve hundredths''. Using that, we can convert to the decimal

$$\boxed{0.12}$$

\end{document}
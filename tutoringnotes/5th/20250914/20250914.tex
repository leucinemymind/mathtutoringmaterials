\documentclass[a4paper, 12pt]{article}
\usepackage{graphicx} % Required for inserting images
\usepackage{fullpage}
\usepackage{amsmath}
\usepackage{xcolor}
\usepackage{float}
\usepackage{geometry}
\usepackage{biblatex}
\geometry{margin=1in}
\usepackage{enumitem}
\usepackage{hyperref}
\usepackage{microtype}
\usepackage{parskip}

\hypersetup{
    colorlinks=true,        % Enable colored links
    linkcolor=teal,         % Set color for internal links
    citecolor=teal,         % Set color for citations
    filecolor=teal,         % Set color for file links
    urlcolor=teal           % Set color for URLs
}

\title{09-14-2025 Positive and negative numbers}
\author{C\&L Math Tutoring}
\date{September 2025}

\begin{document}

\maketitle

\subsection*{Positive numbers vs. negative numbers}

\textbf{Positive numbers} are numbers greater than zero.

\textbf{Negative numbers are numbers less than zero.}

They are represented on the number line with positive numbers to the right of zero and negative numbers to the left.

\subsection*{Absolute value}

\textbf{Absolute value} is the \textit{distance} from a number to 0.

\textit{Distances} are always \textit{positive}, so \underline{absolute value is always positive.}

The greater the absolute value for \textit{positive numbers}, the \textit{greater} the number.

The greater the absolute value for \textit{negative numbers}, the \textit{smaller} the number.

\textcolor{blue}{\textbf{Example:} There are 3 numbers in a set: 50, $-$30, and $-$10. Find each of their absolute values. Which is the greatest? Which is the smallest? Why?}

\textcolor{blue}{\textbf{Solution:} The absolute values are 50, 30, and 10. The greatest number is 50, because that is the only positive number in the set. The smallest number is $-$30, because even though 30 is greater than 10, the value of a number decreases as you move to the left, and $-$30 is to the left of $-$10.}

\subsection*{Adding and subtracting positive numbers}

When adding two positive numbers, the result is always positive.

$$ 5 + 6 = 11 $$

When subtracting a smaller positive number from a larger positive number, the result is positive.

$$ 5 - 3 = 2 $$

When subtracting a larger positive number from a smaller positive number, the result is negative.

$$ 3 - 5 = -2 $$

To subtract a larger positive number from a smaller positive number, first subtract the smaller number from the larger number to get the absolute value, then change the sign to a negative.

\textcolor{blue}{\textbf{Example:} What is $12 - 24$?}

\textcolor{blue}{\textbf{Solution:}  Rearrange...}

$$ 24 - 12 = 12 $$

\textcolor{blue}{Then change the sign to negative.}

$$ =\boxed{-12} $$

\subsection*{Adding and subtracting negative numbers}

When adding two negative numbers, the result is always negative.\footnote{By convention, we put parentheses () around negative numbers after an operator to avoid confusion.}

$$ -5 + (-6) = -5 - 6 = -11 $$

When subtracting a smaller negative number from a larger negative number, the result is negative.

$$ -5 - (-3) = -2 \quad \text{Two negatives make a positive.}$$

When subtracting a larger negative number from a smaller negative number, the result is positive.

$$ -3 - (-5) = -3 + 5 = 2 $$

\subsection*{Multiplication and division}

When multiplying or dividing two numbers of the same sign, the result is positive.

\begin{gather*}
5 \times 3 = 15 \\
-5 \times (-3) = 15
\end{gather*}

When multiplying or dividing two numbers of different signs, the result is negative.

\begin{gather*}
5 \times (-3) = -15 \\
-5 \times 3 = -15
\end{gather*}

\subsection*{Notation is everything!!!}

Now is the time to begin practicing good, clean notation. Double check that you have the right signs before moving onto the next step, and try not to skip any steps.

\newpage

\subsection*{Practice}

Evaluate the following:

\begin{enumerate}[label=\alph*)]
\item $7 + (-3)$
\item $-8 + 5$
\item $-4 - (-6)$
\item $3 - 7$
\item $-2 \times 9$
\item $-12 \div (-4)$
\item $6 \times (-3)$
\item $-15 \div 3$
\item $-5 - 10$
\item $10 - (-2)$
\item $-7 + (-8)$
\item $-4 \times (-9)$
\item $14 \div (-2)$
\end{enumerate}

When you're finished, check your answers on the next page.

\newpage

\subsection*{Answers}

\begin{enumerate}[label=\alph*)]
\item 4
\item $-3$
\item 2
\item $-4$
\item $-18$
\item 3
\item $-18$
\item $-5$
\item $-15$
\item 12
\item $-15$
\item 36
\item $-7$
\end{enumerate}

\newpage

\end{document}
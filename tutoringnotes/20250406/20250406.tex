\documentclass[a4paper, 12pt]{article}
\usepackage{graphicx} % Required for inserting images
\usepackage{fullpage}
\usepackage{amsmath}
\usepackage{xcolor}
\usepackage{float}
\usepackage{geometry}
\usepackage{biblatex}
\geometry{margin=1in}
\usepackage{enumitem}
\usepackage{hyperref}
\usepackage{microtype}
\usepackage{parskip}

\hypersetup{
    colorlinks=true,        % Enable colored links
    linkcolor=teal,         % Set color for internal links
    citecolor=teal,         % Set color for citations
    filecolor=teal,         % Set color for file links
    urlcolor=teal           % Set color for URLs
}

\title{04-06-25 Ratios}
\author{C\&L Math Tutoring}
\date{April 2025}

\begin{document}

\maketitle

\subsection*{What is a ratio?}
A \textbf{ratio} compares the size of one quantity to another. Here are some common ways of expressing ratios:

$$\frac{1}{3} \: \text{(as a fraction)}$$
$$1:3 \: \text{(separated by a colon)}$$
$$1 \: \text{to} \: 3 \: \text{(with words)}$$

\subsection*{Simplifying ratios}
Ratios, like fractions, can be simplified. Treat the first number as the numerator and the second as the denominator.

Example: Simplify $10:36$.

Solution: Simplify as if it were a fraction.

$$10:36 = (10 \div 2) : (36 \div 2) = \boxed{5 : 18}$$

\subsection*{Ratio tables}
Ratio table for $1:5$

\begin{table}[H]
\centering
  \begin{tabular}{|c|c|}
  \hline
  1 & $1 \times 5 = 5$ \\\hline
  2 & $ 2 \times 5 = 10$ \\\hline
  3 & $3 \times 5 = 15$ \\\hline
  4 & $4 \times 5 = 20$ \\\hline
  5 & $ 5 \times 5 = 25$ \\\hline
  6 & $ 6 \times 5 = 30$ \\\hline
  7 & $ 7 \times 5 = 35$ \\\hline
  \end{tabular}
\end{table}

$$ 1:5 = 2:10 = 3:15 = 4:20 = 5:25 = 6:30 = 7:35 $$

\end{document}

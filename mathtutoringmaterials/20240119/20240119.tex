\documentclass[a4paper, 12pt]{article}
\usepackage{graphicx} % Required for inserting images
\usepackage{fullpage}
\usepackage{amsmath}
\usepackage{xcolor}
\usepackage{float}
\usepackage{geometry}
\usepackage{biblatex}
\geometry{margin=1in}
\usepackage{enumitem}
\usepackage{hyperref}
\usepackage{microtype}
\usepackage{parskip}

\hypersetup{
    colorlinks=true,        % Enable colored links
    linkcolor=teal,         % Set color for internal links
    citecolor=teal,         % Set color for citations
    filecolor=teal,         % Set color for file links
    urlcolor=teal           % Set color for URLs
}

\title{1-19-25 Simplifying Fractions}
\author{Emily Zhang}
\date{January 2025}

\begin{document}

\maketitle

\subsection*{Greatest Common Factor Review}
The \textbf{greatest common factor} (GCF) of a group of numbers is the largest number that each number in the group is divisible by. For example, the GCF of 12, 24, and 6 is 6, because 6 is the largest number that 12, 24, and 6 are all divisible by.

Given this information, can you find the GCF of 5, 10, and 15?

Answer: 5

\subsection*{Simplifying Fractions}
\textbf{Simplifying} fractions, also known \textbf{reducing} the fraction to lowest terms among other names, means dividing both the numerator and the denominator by their greatest common factor so that the numerator and denominator will have no common factors except for 1.

Example: Simplify $\frac{12}{36}$.

\textcolor{blue}{\textbf{Step 1:} Find the GCF} of the numerator and the denominator. The GCF in this case is 12.

\textcolor{blue}{\textbf{Step 2:} Divide both the top and the bottom} by the GCF to get $\boxed{\frac{1}{3}}$.

\textcolor{red}{\textbf{Note:} Although the fractions are technically equivalent, reducing the fraction to lowest terms is mathematical convention and you will often have points taken off if you don't simplify completely.}

\href{https://www.youtube.com/watch?v=U-1KjlJAA6M}{Simplest form song}

\href{https://www.youtube.com/watch?v=362JVVvgYPE}{Scratch Garden fractions video}

\end{document}
